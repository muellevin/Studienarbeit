%!TEX root = ../Studienarbeit.tex

\addchap{Abkürzungsverzeichnis}
%nur verwendete Akronyme werden letztlich im Abkürzungsverzeichnis des Dokuments angezeigt
%Verwendung: 
%		\ac{Abk.}   --> fügt die Abkürzung ein, beim ersten Aufruf wird zusätzlich automatisch die ausgeschriebene Version davor eingefügt bzw. in einer Fußnote (hierfür muss in header.tex \usepackage[printonlyused,footnote]{acronym} stehen) dargestellt
%		\acs{Abk.}   -->  fügt die Abkürzung ein
%		\acf{Abk.}   --> fügt die Abkürzung UND die Erklärung ein
%		\acl{Abk.}   --> fügt nur die Erklärung ein
%		\acp{Abk.}  --> gibt Plural aus (angefügtes 's'); das zusätzliche 'p' funktioniert auch bei obigen Befehlen
%	siehe auch: http://golatex.de/wiki/%5Cacronym
%	
\begin{acronym}[YTMMM]

%a

%b
\acro{BB}{Bounding-Box}

%c
\acro{CSI}{Camera Serial Interface}
\acro{CNN}{Convutional Neural Network}
\acro{CM}{Compute Module}

%d
\acro{DNN}{Deep Neural Network}

%e
\acro{eMMC}{Embedded Multi Media Card}

%f
\acro{FFI}{Foreign-Function-Interface}

%g
\acro{GPIO}{General Purpose Input/Output}
\acro{GPU}{Graphics Processing Unit}

%h

%i

%j

%k

%l

%m
\acro{mAP}{mean Average Precision}
\acro{ML}{Machine Learning}

%n
\acro{NPU}{Neural Processing Unit}

%o

%p
\acro{PAP}{Programmablaufplan}
\acro{PWM}{Pulsweitenmodulation}

%q

%r
\acro{R-CNN}{Region-based Convolutional Neural Network}
\acro{RPi}{Raspberry Pi}
\acro{RPN}{Region-Proposal-Network}

%s
\acro{SATA}{Serial AT Attachment}
\acro{SVM}{Support-Vector-Machine}
\acro{SSD}{Single-Shot-Detection}

%t

%u

%v

%w

%x

%y
\acro{YOLO}{You-Only-Look-Once}

%z

\end{acronym}
