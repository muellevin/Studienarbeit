%!TEX root = ../Studienarbeit.tex

\pagestyle{empty}



% \begin{otherlanguage}{english} % auskommentieren, wenn Abstract auf Deutsch sein soll
\begin{abstract}
    The thesis focuses on the development of a portable, efficient, and self-sufficient deterrent system against unwanted small animals. The work encompasses three main aspects: the construction of the self-sufficient system, the housing of the system, and the detection of the target animals.
    \\
    The primary focus of the thesis lies in the real-time detection and localization of the animals. Object detection and stereo vision have been combined to achieve this goal. Various approaches have been tested for object detection, resulting in successful detection of unwanted small animals. The integration of object recognition and stereo vision also enables the recognition of animals in three-dimensional space. Based on this evaluation, a designed "`water cannon'" targeting system can effectively deter animals. With the shock of getting confronted with deterrent System the animals should not come again.
    \\
    However, there are limitations in achieving real-time detection due to the hardware constraints of the system.
\end{abstract}
% \end{otherlanguage} % auskommentieren, wenn Abstract auf Deutsch sein soll
% }

% Die Arbeit beschäftigt sich damit ein portables, effizientes und autarkes Abschrecksystem gegen unliebsame Kleintiere zu entwickeln. Dabei sind drei Schwerpunkte der Arbeit herausgearbeitet worden: der Aufbau des autarken Systems, die unterbringung des Systems und die Erkennung der Tiere, die abgeschreckt werden sollen.
% \\
% Das Kernelement der Arbeit liegt allerdings in der Echtzeiterkennung und Lokalisierung der Tiere. Hierfür wurde Objekterkennung und Stereo-Vision vereinigt. Für die Objekterkennung sind verschiedene Herangehensweisen erprobt worden. Mit den gewonnen Erkenntnissen ist daher ein Erkennung ungewollter Kleintiere möglich gewesen. Die Kombination der Objekterkennung und Stereo-Vision ermöglicht zudem eine Erkennung der Tiere im dreidimensionalen Raum. Aus dieser Auswertung kann dann ein "`Wasserwerfer'"-Zielsystem verwendet werden um die Tiere effektiv abzuschrecken und aus den heimischen Garten zu vertreiben.
% \\
% Einschränkungen gibt es dabei allerdings in der Echtzeiterkennung, da sie mit der verwendeten Hardware nicht so einfach zu realisieren ist.

% \iflang{en}{%
% % Dieser englische Teil wird nur angezeigt, wenn die Sprache auf Englisch eingestellt ist.
% \renewcommand{\abstractname}{\langabstract} % Text für Überschrift

% \begin{abstract}
% An abstract is a brief summary of a research article, thesis, review, conference proceeding or any in-depth analysis of a particular subject or discipline, and is often used to help the reader quickly ascertain the paper's purpose. When used, an abstract always appears at the beginning of a manuscript, acting as the point-of-entry for any given scientific paper or patent application. Abstracting and indexing services for various academic disciplines are aimed at compiling a body of literature for that particular subject.

% The terms précis or synopsis are used in some publications to refer to the same thing that other publications might call an ``abstract''. In ``management'' reports, an executive summary usually contains more information (and often more sensitive information) than the abstract does.

% Quelle: \url{http://en.wikipedia.org/wiki/Abstract_(summary)}

% \end{abstract}
% }