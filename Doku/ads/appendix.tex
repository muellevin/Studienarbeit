% !TeX root = ../Studienarbeit.tex

\addchap{Anhanng}
\appendix

{\let\clearpage\relax  \chapter{Arduino UNO Testcode}} \label{an:ard}

\begin{lstlisting}[label=list:Arduino_Test,caption=Arduino Test Code um alle Funktionen zu überprüfen.]
    #include <Servo.h>
    
    int horizontalServoPin = 10;
    Servo horizontalServo;
    float horizontalServoPos = 90;
    float horizontalServoStepSize = 20;
    
    int verticalServoPin = 9;
    Servo verticalServo;
    float verticalServoPos = 90;
    float verticalServoStepSize = 10;
    
    int powerServoPin = 7;
    
    int powerLightPin = 6;
    
    int powerPumpPin = 4;
    
    int soundSignalPin = 3;
    unsigned long currentSoundFrequency = 15000;
    unsigned long UpperSoundFrequency = 44000;
    unsigned long lowerSoundFrequency = 15000;
    int frequencyStepSize = 100;
    
    int powerSoundPin = 2;
    
    void setup() {
      // put your setup code here, to run once:
      Serial.begin(9600);
      horizontalServo.attach(horizontalServoPin);
      verticalServo.attach(verticalServoPin);
      pinMode(powerServoPin, OUTPUT);
      pinMode(powerLightPin, OUTPUT);
      pinMode(powerPumpPin, OUTPUT);
      pinMode(powerSoundPin, OUTPUT);
      pinMode(soundSignalPin, OUTPUT);
      digitalWrite(powerServoPin, HIGH);
    }
    
    void loop() {
      if (millis() % 500 == 0) {
        moveSevo(horizontalServo, 0, 180, &horizontalServoStepSize, &horizontalServoPos);
        Serial.print("Moving horizontal Servo to position ");
        Serial.println(horizontalServoPos); // Printing the position of the horizontal servo
      }
      if (millis() % 1000 == 0) {
        moveSevo(verticalServo, 45, 135, &verticalServoStepSize, &verticalServoPos);
        Serial.print("Moving vertical Servo to position ");
        Serial.println(verticalServoPos); // Printing the position of the vertical servo
      }
      if (millis() % 200 == 0) {
        digitalWrite(powerLightPin, !digitalRead(powerLightPin));
        Serial.println("Toggling powerLightPin."); // A message indicating powerLightPin has been toggled
      }
      if (millis() % 5000 == 0) {
        digitalWrite(powerServoPin, !digitalRead(powerServoPin));
        Serial.println("Toggling powerServoPin."); // A message indicating powerServoPin has been toggled
      }
      if (millis() % 1000 == 0) {
        digitalWrite(powerPumpPin, !digitalRead(powerPumpPin));
        Serial.println("Toggling powerPumpPin."); // A message indicating powerPumpPin has been toggled
      }
      if (millis() % 10000 == 0) {
        digitalWrite(powerSoundPin, !digitalRead(powerSoundPin));
        Serial.println("Toggling powerSoundPin."); // A message indicating powerSoundPin has been toggled
      }
      if (millis() % 100 == 0) {
        if (currentSoundFrequency < lowerSoundFrequency) {
          currentSoundFrequency = lowerSoundFrequency;
          frequencyStepSize *= -1;
        } else if (currentSoundFrequency > UpperSoundFrequency) {
          currentSoundFrequency = UpperSoundFrequency;
          frequencyStepSize *= -1;
        }
        tone(soundSignalPin, currentSoundFrequency);
        currentSoundFrequency += frequencyStepSize;
        Serial.print("Playing sound with frequency: ");
        Serial.println(currentSoundFrequency); // Printing the frequency of the sound played
      }
    }
    
    
    void moveSevo(Servo motor, uint8_t minPos, uint8_t maxPos, float* stepSize, float* currentPosition) {
      if (*currentPosition < minPos) {
        *currentPosition = minPos;
        *stepSize *= -1;
      } else if (*currentPosition > maxPos) {
        *currentPosition = maxPos;
        *stepSize *= -1;
      }
      motor.write(*currentPosition);
      *currentPosition += *stepSize;
    }
    
    \end{lstlisting}
