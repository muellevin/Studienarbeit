%!TEX root = ../Studienarbeit.tex

\section{Stand der Technik}

Um unliebsame Besucher aus dem Garten, Haus oder Auto zu vertreiben gibt es viele Geräte auf dem Markt. Zu diesen gehört ein großes Sortiment von Ultraschall-Tierschreck-Systemen, Sprinkleranlagen und verschieden Varianten von Weidezäunen. Um einen bestmöglichen Erfolg der Vertreibung zu bieten, sollen die Geräte an den Umschlagsorten der Tiere platziert werden.

Die einzelnen Systeme und deren Vor- und Nachteile werden in den nachfolgenden Unterkapiteln beschrieben.



\subsection{Ultraschall-Tierschreck} \label{ton_schreck}

Eine gängige Variante des Ultraschall-Tierschrecks ist der Marderschreck. Der Marderschreck wird in dem Motorraum eines Fahrzeugs platziert und soll verhindern, dass der Marder Schläuche und Kabel durchbeißt. Er verspricht das Fernhalten und Vertreiben der Tiere durch aussenden eines Hochfrequenztons. Der Ton hat dabei eine Frequenz von 17 bis 45 kHz. Für die Tiere ist dieser Frequenzbereich besonders unangenehm. Erwachsene Menschen nehmen diese Töne aber kaum bis gar nicht wahr. \cite{marderschreck}

Daher werden auch für den heimischen Garten diese Geräte gerne eingesetzt. Da sie aber nicht länger über die Autobatterie und der Lichtmaschine mit Energie versorgt werden, sind sie häufig an eine kleine Batterie und einem Solarpanel angeschlossen. Die gängigen Varianten eines Ultraschall-Tierschrecks für die Gartenverwendung haben zudem ein eingebautes Blitzlicht. Bei Nacht wird das Tier durch kurze Lichtimpulse zusätzlich beim Durchqueren des Gartens gestört und der Erfolg zur Vertreibung von nachtaktiven Tieren erhöht sich.
\\
Die Tiere, insbesondere Waschbären, gewöhnen sich allerdings an das Licht und dem Hochfrequenzton. Daher hält der Erfolg der Vergrämung oft nur wenige Wochen an.
Außerdem können die Geräte auch Probleme bereiten. Die eigenen Haustiere und auch Kleinkinder nehmen den Hochfrequenzton ebenso wahr. Da die Geräte auf jegliche Bewegung reagieren, kann es sein das der Ultraschall-Tierschreck vor Betreten des Gartens für die Haustiere und Kinder deaktiviert werden müssen. \cite{anti_wasch}


\subsection{Automatische Sprinkleranlage}

Eine andere Variante von Abschrecksystem ist der Einsatz von Sprinkleranlagen. Durch die beschießen mit Wasser werden die Tiere in besonders gut gestört. Der automatische Sprinkler wird über einen Bewegungsmelder ausgelöst und versprüht großflächig Wasser im Zielbereich. Nach eigener Erfahrung hat eine automatische Sprinkleranlage eine höhere Erfolgsquote, um ungewollte Besucher aus dem heimischen Garten zu vertreiben. \cite{anti_wasch}\\
Der Nachteil bei diesem System ist es, dass der Sprinkler direkt mit einem Gartenschlauch verbunden werden muss. Dadurch treten deutliche Einschränkungen in der Positionierung des Abschrecksystems auf, da ein Wasseranschluss mit ausreichend Druck in der Nähe sein muss. Zusätzlich fällt der Druck mit zunehmender Länge des Gartenschlauches ab, wodurch der Zielbereich weiter eingeschränkt wird. Zum Beispiel:

\comment{Beispiel/Begründung Strömungslehre}

Da die Geräte, wie der Ultraschall-Tierschreck, über einen Bewegungsmelder ausgelöst werden, tritt ein weiteres Problem auf. Auch der Mensch löst den Sprinkler unbeabsichtigt aus. Im Gegensatz zum Ultraschall-Tierschreck kann dies nicht nur unangenehm, sondern sehr ärgerlich werden.

Ein anderer Punkt, der bei diesen Anlagen häufig vernachlässigt oder nicht vermeidbar ist, ist die Verschwendung von Trinkwasser. Vor allem in Zeiten des Energie- und Wassersparens versucht man Verschwendungen zu minimieren. Einige Landkreise sind in den letzten Jahren sogar so weit gegangen, dass das Rasensprengen aus eigener Quelle von 12 bis 18 Uhr verboten ist. Durch diese Maßnahmen erhofft man sich besser durch Dürreperioden zu kommen. Ein automatischer Rasensprinkler, der Trinkwasser verwendet sollte daher vermieden werden. \cite{wasser_verbot}

\subsection{Weidezaun}

Der Weidezaun ist häufig das letzte Mittel um die Tiere aus dem Garten zu bekommen. Der Zaun wird um den Garten herum aufgebaut, so dass sämtliche Durchgänge, an denen die Tiere in den Garten eindringen können, blockiert werden. Wenn das Getier nun versucht an durch diese Zugänge in den Garten einzudringen, wird der Zaun einen schwachen elektrischen Schlag abgeben.
\\
Ein Waschbär wird durch diesen Schlag es nicht noch einmal versuchen den Zugang zu verwenden. Häufig suchen die Tiere stattdessen einen anderen Zugang in den Garten. Wenn die Tiere keinen anderen Zugang zum Garten finden, wenden sie sich zunächst vom Grundstück ab. In unregelmäßigen Abständen überprüfen die Tiere allerdings ob die Blockade immer noch besteht. Der Weidezaun muss daher ständig eingeschaltet und gewartet werden. \cite{anti_wasch}

Der Weidezaun hat von den genannten Systemen die höchste Effektivität, wenn es um die Vertreibung der Tiere geht. Durch die Blockierung sämtlicher Zugänge besteht allerdings ein sehr hoher Installationsaufwand. Zusätzlich müssen weiter Elemente zum Weidezaun installiert werden, da auch die Zugänge für den Menschen blockiert sind. Entsprechende Gartentore sind anzuschaffen. Das gesamte System wird somit schnell teuer, weshalb er auch als letztes Mittel betrachtet wird.


\section{Computer Vision}

Befähigung Computerprogramme "`sehen"' zu lassen.

\subsection{Vision}

Bilderkennungsmerkmale etc.

\subsection{Stereo Vision}

Tiefenwahrnehmung für anständiges Zielen (statt ML? und zu viele Tricks?)

\section{Dreidimensionales Zielsystem}

Mechanisches System beschreiben etc.

\subsection{Strömungslehre}

Pumpe $\rightarrow$ Nötigung anständiger Berechnung des Zielsystem

\subsection{Tiefenberechnung}

Zusätzlich zu betrachtende Effekte Luftwiderstand; Dispersion; Wind/Error-correction etc.
Bezug Error: Bilderkennung des Wasserstrahls $\rightarrow$ Regelungstechnik und Autokalibrierung
+ evtl. Einspielen der Tiefenwahrnehmung an neuer Position (nicht bei Stereo-Vision nötig).
