%!TEX root = ../Studienarbeit.tex

\section{Stand der Technik in der Tiervertreibung}

Um unliebsame Besucher aus dem Garten, Haus oder Auto zu vertreiben gibt es viele Geräte auf dem Markt. Zu diesen gehört ein großes Sortiment von Ultraschall-Tierschreck-Systemen, Sprinkleranlagen und verschieden Varianten von Weidezäunen. Um einen bestmöglichen Erfolg der Vertreibung zu bieten, sollen die Geräte an den Umschlagsorten der Tiere platziert werden.

Die einzelnen Systeme und deren Vor- und Nachteile werden in den nachfolgenden Unterkapiteln beschrieben.

\subsection{Ultraschall-Tierschreck} \label{ton_schreck}

Eine gängige Variante des Ultraschall-Tierschrecks ist der Marderschreck. Der Marderschreck wird in dem Motorraum eines Fahrzeugs platziert und soll verhindern, dass der Marder Schläuche und Kabel durchbeißt. Er verspricht das Fernhalten und Vertreiben der Tiere durch aussenden eines Hochfrequenztons. Der Ton hat dabei eine Frequenz von 17 bis 45 kHz. Für die Tiere ist dieser Frequenzbereich besonders unangenehm. Erwachsene Menschen nehmen diese Töne aber kaum bis gar nicht wahr. \cite{marderschreck}

Daher werden auch für den heimischen Garten diese Geräte gerne eingesetzt. Da sie aber nicht länger über die Autobatterie und der Lichtmaschine mit Energie versorgt werden, sind sie häufig an eine kleine Batterie und einem Solarpanel angeschlossen. Die gängigen Varianten eines Ultraschall-Tierschrecks für die Gartenverwendung haben zudem ein eingebautes Blitzlicht. Bei Nacht wird das Tier durch kurze Lichtimpulse zusätzlich beim Durchqueren des Gartens gestört und der Erfolg zur Vertreibung von nachtaktiven Tieren erhöht sich.
\\
Die Tiere, insbesondere Waschbären, gewöhnen sich allerdings an das Licht und dem Hochfrequenzton. Daher hält der Erfolg der Vergrämung oft nur wenige Wochen an.
Ultraschallgeräte können aber auch Probleme bereiten. Die eigenen Haustiere und auch Kleinkinder nehmen den Hochfrequenzton ebenso wahr. Da die Geräte auf jegliche Bewegung reagieren, kann es sein das der Ultraschall-Tierschreck vor Betreten des Gartens deaktiviert werden muss, damit die Haustiere und Kinder sich im Garten aufhalten können.\cite{anti_wasch}


\subsection{Automatische Sprinkleranlage}

Eine andere Variante von Abschrecksystem ist der Einsatz von Sprinkleranlagen. Durch das Beschießen mit Wasser werden die Tiere besonders gut gestört. Der automatische Sprinkler wird über einen Bewegungsmelder ausgelöst und versprüht großflächig Wasser im Zielbereich. Nach eigener Erfahrung hat eine automatische Sprinkleranlage eine höhere Erfolgsquote, um ungewollte Besucher aus dem heimischen Garten zu vertreiben, aber sie kann durch den Bewegungsmelder auch unbeabsichtigt von einem selbst ausgelöst werden.\cite{anti_wasch}

Der Nachteil bei diesem System ist, dass der Sprinkler direkt mit einem Gartenschlauch verbunden werden muss. Dadurch treten deutliche Einschränkungen in der Positionierung des Abschrecksystems auf, da ein Wasseranschluss mit ausreichend Druck und Volumenstrom an ihm befestigt sein muss. Zusätzlich fällt der Druck und der Volumenstrom mit zunehmender Länge des Gartenschlauches ab, wie in den Artikel \textit{1/2 Zoll vs. 3/4 Zoll Gartenschlauch – Wann benötigt man welchen und welche Unterschiede gibt es?} aufgefallen ist. Der Author hat in diesem Artikel zwei verschieden Gartenschläuche und deren Druck- und Volumenstromverlust gemessen. Bei dem Test des Gartenschlauches mit 1/2 Zoll Durchmesser war nach zwei Metern bereits eine Reduzierung des Volumenstroms auf 62\% zu beobachten.\\

\comment{Selber noch überprüfen und bestätigen}
Angenommen ein eingesetzter Sprinkler hätte eine Düse mit einer Öffnung von 1,5 Millimeter in einer Höhe von einem Meter befestigt, so hätte die Reduzierung den Zielbereich von knappen 10 Metern auf 6 Meter reduziert.
Ein Sprinkler hätte somit mit zunehmender Entfernung proportional zum Volumenstrom an Reichweite verloren.

Ein anderer Punkt, der bei diesen Anlagen häufig vernachlässigt wird, ist die Verschwendung von Trinkwasser. Vor allem in Zeiten des Energie- und Wassersparens versucht man Verschwendungen zu minimieren. Einige Landkreise sind in den letzten Jahren in den Dürreperioden sogar so weit gegangen, dass das Rasensprengen aus eigener Quelle von 12 bis 18 Uhr verboten worden ist. Durch diese Maßnahmen erhofft man sich besser durch Dürreperioden zu kommen. Ein automatischer Rasensprinkler, der Trinkwasser verwendet sollte daher vermieden werden, um Dürreperioden nicht noch schlimmer zu machen.\cite{wasser_verbot}

\subsection{Weidezaun}

Der Weidezaun ist häufig das letzte Mittel um die Tiere aus dem Garten zu bekommen. Der Zaun wird um den Garten herum aufgebaut und sämtliche Durchgänge, an denen die Tiere in den Garten eindringen  können, sollen ebenfalls mit dem Zaun blockiert werden. Wenn das Getier nun versucht durch diese Zugänge in den Garten einzudringen, wird der Eindringling von dem Zaun einen schwachen elektrischen Schlag abbekommen.
\\
Ein Waschbär wird durch diesen Schlag in absehbarer Zeit es nicht noch einmal versuchen denselben Zugang zu verwenden. Häufig suchen die Tiere stattdessen einen anderen Zugang in den Garten. Wenn die Tiere keinen anderen Zugang zum Garten finden, wenden sie sich vom Grundstück ab. In unregelmäßigen Abständen überprüfen die Tiere allerdings ob die Blockade immer noch besteht. Der Weidezaun muss daher ständig eingeschaltet und gewartet werden.\cite{anti_wasch}

Der Weidezaun hat von den genannten Systemen die höchste Effektivität, wenn es um die Vertreibung der Tiere geht. Durch die Blockierung sämtlicher Zugänge besteht allerdings ein sehr hoher Installationsaufwand. Zusätzlich müssen weitere Elemente, wie Gartentore installiert werden, da sonst auch die Zugänge für den Menschen blockiert sind. Das gesamte System wird somit schnell teuer, weshalb der Weidezaun auch als letztes Mittel betrachtet wird.


\section{Computer Vision}

Computer Vision beschäftigt sich damit Computern das \enquote{Sehen} zu ermöglichen. Schon jahrzehntelang versuchen dies Wissenschaftler zu erreichen. Heutzutage sind sie schon sehr weit gekommen. Computer Vision findet in der Logistik, dem autonomen Fahren, Gesichtserkennung und bei noch vielen anderen Bereichen große Zuwendung. Ziel davon ist es Analysen, Verarbeitung und Interpretationen von Bildern und Videos durch einen Computer möglich zu machen, damit eine Maschinen visuelle Informationen so verstehen kann wie wir Menschen es tun. \cite{cv_Szeliski}

\subsection{Object Detection}

Object Detection ist eine Anwendung der Computer Vision, die es ermöglicht, Objekte in einem Bild oder Video zu erkennen und zu lokalisieren. Object Detection-Systeme verwenden Algorithmen, die auf \ac{ML}- und \ac{DNN}-Techniken basieren, um Objekte in Bildern oder Videos zu erkennen und zu klassifizieren. Das erkannte Objekt wird mittels einer \ac{BB} aus den Bild extrahiert.\\
\comment{Bild rein}
Um diese Detektion zu erhalten gibt es verschieden Basen und Architekturen von Object Detections Systemen, die im folgenden beschreiben sind. \cite{cv_Szeliski}

\subsection{Two-Stage-Detectors}

Eins der frühen Modelle die diese Technik verwendet hat ist das \ac{R-CNN}. Es werden hirfür zwei große Schritte angewendet. Weshalb das \ac{R-CNN} und ähnliche Modelle als \textit{two-stage-Detectors} bezeichnet werden.\\
Der erste Schritt bei diesen Modellen ist es mittels eines \ac{RPN} rechteckige Regionen in einen Bild zu bestimmen.
Das \ac{RPN} schlägt hierbei eine Unterteilung des Eingabebildes in verschiedene Regionen vor. Das \ac{R-CNN}-Netzwerk lässt sich dabei ~2000 Regionen vorschlagen.
Aus diesen Regionen werden anschließend im zweiten Schritt die Feautures eines Eingabebildes durch ein \ac{CNN} extrahieren. Im Falle von \ac{R-CNN} werden die Feautures mittels einer Softmax-Schicht und einer \ac{SVM} ausgewertet und Objekte detektiert.\\
Dabei sollte die Anzahl der Regionen vernünftig klein sein, damit das nachfolgende extrahieren und klassifizieren in einer absehbarer Zeit erfolgt. Dieser zusätzliche Zeitaufwand, der für das \ac{RPN} aufgewandt wird, ist zudem auch ein Nachteil gegenüber anderen Ansätzen. Die Two-Stage-Detektoren erhalten dadurch zwar eine höhere Genauigkeit in der Lokalisierung und Klassifizierung, benötigen aber auch mehr Zeit.\cite{R_CNN}

\subsection{One-Stage-Detectors}

One-Stage Detektoren sind einer der neueren Ansätze Object-Detection zu realisieren.
Sie basieren auf die menschliche Natur und gehen von einer Single-Shot Erkennung aus. Dafür wenden sie entgegen den Two-Stage-Detektoren keinen Region-basierten Algorithmus an.
Stattdessen unterteilen sie ein Eingabebild in ein $S \times  S$ großes Gitternetz. Auf den Gitter-Boxen wird daraufhin ein Klassifizierung und Lokalisierung vorgenommen.\\
Dies war auch der erste Ansatz in der \ac{YOLO}-Architektur. Durch diesen Ansatz ist das \ac{YOLO}-V1 Modell entgegen den Two-Stage-Detektor Faster \ac{R-CNN} laut den Autoren in \cite{YOLO_V3} neun mal schneller bei der Detektierung. \ac{YOLO}-V1 ist aber deutlich schlechter, da deutlich weniger Objekte richtig detektiert worden sind. In Version 2 wird daher statt einem Gitternetz die Anchor-Boxen verwendet. Anchor-Boxen haben keine einheitliche Größe. Sie sind rechteckige Boxen, welche in verschiedenen Größen und Seitenverhältnissen angewendet werden. Die Anchor-Boxen werden auf einer Reihe von vordefinierten Positionen über ein Eingabebild verteilt.\\
Bei diesem Ansatz wird die Detektierung durch die Anchor-Boxen ermöglicht, ohne einen Verlust der Geschwindigkeit zu verursachen. Die Version 2 von \ac{YOLO} enthält noch andere Anpassungen, die die Anzahl an richtigen Detektionen weiter erhöht hat.
\cite{YOLO_V3}

Es gibt noch weitere nicht Anchor oder Gitter betriebene Architekturen, wie die \textit{CenterNet} oder \textit{CornerNet}-Architektur. Diese ermitteln den Mittelpunkt eines Objektes oder deren Kanten. Sie erhöhen die Präzession eines Detektors, benötigen aber wiederum mehr Ausführungszeit.\cite{yolo_all}

\subsection{Tiefenberechnung}

Zusätzlich zu betrachtende Effekte Luftwiderstand; Dispersion; Wind/Error-correction etc.
Bezug Error: Bilderkennung des Wasserstrahls $\rightarrow$ Regelungstechnik und Autokalibrierung
+ evtl. Einspielen der Tiefenwahrnehmung an neuer Position (nicht bei Stereo-Vision nötig).

\section{Komponenten}

\subsection{Aktoren}

Mechanisches System beschreiben etc.

\subsection{Strömungslehre}

Pumpe $\rightarrow$ Nötigung anständiger Berechnung des Zielsystem

\begin{equation} \label{v_qa}
    v = Q/A
\end{equation}

\begin{equation}\label{Rohreibung}
    \Delta p = \frac{\lambda \times L \times \rho \times Q^2}{2 \times A^2}
\end{equation}

mit Bernoulligleichung:

\begin{equation}
    \frac{v_{1}^2 \times \rho}{2} + p_1 = \frac{v_{2}^2 \times \rho}{2} + p_2 + \frac{\lambda \times L \times \rho \times Q^2}{2 \times A^2}
\end{equation}

da zunächst bei Hausleitung Druck ca. 4 Bar für $p_1$

\subsection{3D-Druck}
