
%!TEX root = ../Studienarbeit.tex

\section{Bilderkennung und -Verarbeitung}


\section{Hardwarerealisierung}

\comment{Verkabelung, Aufbau, Herausforderungen, spezielle Lösungen}
Auswahl PI -> Liefermangel
Erstidee Multiplexer um zeitverzögerung zu reduzieren -> schlechte Kritik
% https://www.amazon.de/Arducam-Adaptermodul-Raspberry-kompatibel-Kameras/dp/B07TYL36MC/ref=sr_1_5?__mk_de_DE=%C3%85M%C3%85%C5%BD%C3%95%C3%91&crid=2KI1QVYFS8KN6&keywords=camera+adapter+raspberry&qid=1662198136&sprefix=camera+adapetr+raspberry%2Caps%2C77&sr=8-5

Zweite Idee usb-> von vornerein "komplexer" /durch zeitkacke -> irgendwie auch teuer

Dritte Idee esp32-CAM Module -> Hohe Latenzen und Zeitsynchronitätsaufwand

Viertens -> compute model + I/O shield auch teuer aber 2 csi Anschlüsse

5. Jetson mit 2 CSI Anschlüssen -> teuer und noch mehr Liefermangel

\subsection{Wasserversorgung}

integrierte Pumpe $\rightarrow$ Soll portable sein

Alternativ: Gartenschlauch mit Ventilsteuerung

\subsubsection{Pumpe}

Orientierung Gartenschlauch/Sprinkler:
Durchfluss in etwa 20L/min; Druck bis 4 Bar; Düsenspritze ca. 1-2mm Durchmesser

Pumpe: Membranpumpe $\rightarrow$ Gleichbleibende Fördermenge
bei hohen Druckunterschieden (Pumpe 1-4 Bar).
Druck Vernachlässigen und Strömungslehre mit Fördermenge
berechnen. Membranpumpe haben Druckschalter $\rightarrow$
wie ist ein/ausschaltverfahren? Delay etc. Auswahl Verwendung
von Ventil zum Durchschalten oder von Versorgungsspannung der Pumpe???

Normale Pumpe: Komplexer durch kennlinie $\rightarrow$ Berechnung des
inneren Drucks nötig und Interpolation dieser. (Meiste Pumpen
haben schon bei 1-2 Bar Probleme $\rightarrow$ Druckventil schaltet Pumpe aus.

\subsubsection{Wassertank}
Integriert oder Schlauch zu extern. Vor-/Nachteile Entscheidung?

Integriert:

Vollständig portable

;schwerer, Dichtigkeitsproblem

Extern:

Bedingt portable, einfacher zu realisieren, Systembetrachtung geringer

Wasserversorgung muss am Einsatzort möglich sein, Lange Strecken und Höhen für
Pumpe nicht gut.


\section{Dreidimensionales Zielsystem}

\comment{Mit Vision Schwerpunkt der Arbeit teils/ganz in HW-Realisierung?}
