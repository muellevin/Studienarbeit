%!TEX root = ../Studienarbeit.tex

\section{Begrifflichkeiten und Systeme}

\subsection{Rust}

"`Rust ist einer der beliebtesten Programmiersprachen und das 7 Jahre in Folge."'-\cite{stack_rust}

Einer der Gründe dafür ist, dass Rust sich von dynamisch-typisierte
Sprachen inspirieren lassen hat. Durch diese Inspiration
ist es nicht nötig
Variablen mit konkreten Datentypen zu deklarieren. Bei der
Kompilierung werden diese automatisch erkannt und verwendetet.
Rust setzt allerdings ein stark typisiertes System ein.
Bei der Kompilierung werden die dynamischen Datentypen aufgelöst
und Datentypfehler direkt erkannt.
Dadurch werden viele Datentypkonversionsfehler während der Kompilierung und somit
in der Entwicklungsphase aufgedeckt. Ohne Erkennung führen
sie zu Problemen in der Laufzeit eines Programms.
Der Compiler verfügt zugleich über umfangreiche
Checks, die vor weiteren Laufzeitfehlern schützen sollen.
Ohne diese Checks muss ein Entwickler diese mittels eigener Tests finden und abfangen.
\\
Dem Entwickler wird zusätzlich die Möglichkeit geboten zu entscheiden
ob Daten auf dem Stack oder dem Heap
gespeichert werden sollen. Durch eine bewusste Nutzung
dieser Möglichkeit kann noch das letzte Bisschen
Performanz aus dem Speichermanagement geholt werden.
\\
Ein großer Vorteil entgegen C\verb|#|, C/C++ oder Java liegt darin, dass
der Compiler erkennen kann, wann Daten nicht länger benötigt werden
und deren belegten Speicher automatisch freigibt. Dies ermöglicht ein effizientes
Nutzen des Speichers ohne viel Aufwand dem Entwickler aufzubürden.
Vergleichsweise viel Aufwand hat man in C/C++. Dort müssen
verschiedene Speichernutzungen mit \emph{malloc} und \emph{free} manuell gemanagt.
\\
Mit diesen Möglichkeiten kann Rust direkten \emph{low-level} Code mit
hohen Sicherheitsgarantien produzieren.
Da Rust noch jung ist, sind Neuentwicklungen
von Desktopanwendungen schwer umsetzbar. Durch zu wenige
Features für die Entwicklung graphischer Oberflächen sind diese
nur schwer zu realisieren. Dadurch
werden häufig Performanz- und sicherheitskritische
Abschnitte in anderen Sprachen mittels
dem \acf{FFI} ausgetauscht. \cite{rust_overflow_blog}


\subsection{Raspberry Pi}

\subsection{Computer Vision-Stereo Vision}

\section{Problemstellung}
